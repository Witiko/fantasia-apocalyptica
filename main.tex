\RequirePackage{luatex85}
\PassOptionsToPackage{shorthands=off}{babel}
\makeatletter
\disable@package@load{fontenc}
\makeatother
\let\oldlooseness=\looseness
\documentclass{csbulletin}
\selectlanguage{czech}
\setcounter{secnumdepth}{3}
\usepackage{luavlna}
\usepackage[strict]{lua-widow-control}
\usepackage{csquotes}
\usepackage[
  backend=biber,
  style=iso-numeric,
  sortlocale=cs,
  autolang=other,
  bibencoding=UTF8,
  mincitenames=2,
  maxcitenames=2,
]{biblatex}
\addbibresource{main.bib}
\usepackage{minted}
\usemintedstyle{bw}
\setminted{firstnumber=last, linenos}
\newenvironment{mintedblock}{%
  \par\vspace{\topsep}\vspace{\partopsep}%
  \begingroup
  \fvset{listparameters=\setlength{\topsep}{0pt}\setlength{\partopsep}{0pt}}%
}{%
  \endgroup
  \par\vspace{\topsep}\vspace{\partopsep}%
}
\usepackage[
  implicit=false,
  hidelinks,
]{hyperref}
\newcommand\vref[1]{\ref{#1} na straně~\pageref{#1}}

\newcommand\pkg{\textsf}
\newcommand\acro[1]{\textsc{\MakeLowercase{#1}}}

\begin{document}

\singlechars{czech}{AaIiVvOoUuSsZzKk}

\title{Příprava videozáznamu české premiéry multimediálního díla ,,Fantasia Apocalyptica``}
\EnglishTitle{Recording the Czech premiere of the ``Fantasia Apocalyptica'' multimedia work}
\author{Vít Starý Novotný}
\podpis{Vít Starý Novotný, witiko@mail.muni.cz}
\maketitle[1.5ex]

\begin{abstract}
11. října 2019 se při příležitosti oslav 25. výročí založení Fakulty informatiky Masarykovy univerzity uskutečnila česká premiéra multimediálního díla \emph{Fantasia Apocalyptica} za osobní účasti autora, Donalda Knutha. Z představení byl pořízen videozáznam, který je veřejně dostupný na YouTube kanálu Fakulty informatiky.

V tomto článku popisuji přípravu záznamu od akvizice a zpracování surových audiovizuálních dat přes přípravu replik panelů doprovázejících představení po spojení jednotlivých částí do výsledného záznamu. S výjimkou střihu úvodního slova a závěru proběhlo veškeré zpracování pomocí svobodných nástrojů jako \TeX, Audacity, FFmpeg, \acro{MLT} a ImageMagick. V článku ukazuji, jak může čtenář tyto nástroje použít při přípravě vlastních záznamů.
\end{abstract}
\klicovaslova: audiovizuální záznam, zpracování videa, titulky, zpracování zvuku, kinetická typografie, \TeX, Audacity, FFMpeg, \acro{MLT}, ImageMagick, \acro{ASS}

\bigskip

11. října 2019 se v jezuitském kostele Nanebevzetí Panny Marie uskutečnila česká premiéra multimediálního díla \emph{Fantasia Apocalyptica}~\cite{knuth2023fantasia} za osobní účasti autora, Donalda Knutha. Z představení byl pořízen videozáznam, který je veřejně dostupný na YouTube kanálu Fakulty informatiky Masarykovy univerzity (\acro{FI~MU})~\cite{fimu2020czech}. Předchozí články ve Zpravodaji \CSTUG u popisují průběh představení~\cite{luptak2019fantasia} a přednášek Donalda Knutha při příležitosti návštěvy Brna~\cite{szaniszlo2020dva}. V tomto článku popisuji přípravu záznamu představení.

Nejprve v sekci~\ref{sec:akvizice} popisuji akvizici a zpracování surových audiovizuálních dat. Následně v sekci~\vref{sec:uvodni-slovo-a-zaver} popisuji přípravu videozáznamu úvodního slova a závěru představení. Dále v sekci~\vref{sec:panely} popisuji přípravu panelů doprovázejících představení. Nakonec v sekci~\vref{sec:spojeni} popisuji spojení jednotlivých částí do výsledného záznamu. V sekci \vref{sec:zaver} shrnuji poznatky z článku a popisuji promítání hotového videozáznamu na \acro{FI~MU} 18. prosince 2019.

\section{Akvizice a zpracování videa a zvuku}
\label{sec:akvizice}

TODO

\subsection{Záznam videa}

TODO

\subsection{Zpracování videa}

TODO

\subsection{Záznam zvuku}

TODO

\subsection{Zpracování zvuku}

TODO

\section{Příprava videozáznamu úvodního slova a závěru}
\label{sec:uvodni-slovo-a-zaver}

TODO

\subsection{Střih úvodního slova a závěru}

TODO

\subsection{Titulky úvodního slova}

TODO

\section{Příprava panelů doprovázejících varhanní oratorium}
\label{sec:panely}

TODO

\subsection{Příprava panelu s trojjazyčným textem Zjevení Janova}

TODO

\subsection{Příprava panelu s ilustracemi Duanea Bibbyho}

TODO

\subsection{Příprava panelu s notami pro varhany}

TODO

\section{Spojení jednotlivých částí do výsledného videa}
\label{sec:spojeni}

TODO

\section{Závěr}
\label{sec:zaver}

TODO

\printbibliography

\begin{summary}
On October 11, 2019, the Czech premiere of \emph{Fantasia Apocalyptica} was held for the 25th anniversary of Masaryk University's Faculty of Informatics, featuring its author, Donald Knuth. A video recording of the perfomance was taken and published at the YouTube channel of the Faculty of Informatics.

In this article, I describe how the recording was prepared from processing the raw footage through replicating the panels that accompanied the performance to the final composition. Apart from the intro and wrap-up, only free open-source tools like \TeX, Audacity, FFMpeg, \acro{MLT}, and ImageMagick were used. In the article, I describe how readers can use these tools to create their own recordings.

\keywords: footage acquisition, audio-visual production, subtitles, kinetic typography, \TeX, Audacity, FFMpeg, \acro{MLT}, ImageMagick, \acro{ASS}
\end{summary}

\end{document}