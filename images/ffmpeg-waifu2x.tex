\begin{figure}[H]
\input images/vidstabdetect-vidstabtransform
\caption{Stabilizace videa pomocí filtrů nástroje FFmpeg.}
\label{fig:vidstabdetect-vidstabtransform}
\end{figure}%

\begin{figure}[b!]
\input images/waifu2x.tex
\caption{Navýšení rozlišení videa pomocí nástroje waifu2x.}
\label{fig:waifu2x}
\end{figure}%

\begin{figure}[H]
\input images/minterpolate-yadif
\caption{Navýšení snímkové frekvence videa pomocí filtrů nástroje FFmpeg.}
\label{fig:minterpolate-yadif}
\end{figure}%
\begin{filecontents}[overwrite,nosearch,noheader]{output-parameters.sh}
-c:v libx265 -crf 17 -preset placebo -pix_fmt yuv420p -c:a aac -b:a 384k -ar 44.1k
\end{filecontents}
\addtocounter{footnote}{-1}
\ClassWarning{csbulletin}{Make sure that the footnote text is on the left-hand page and the footnote mark is on the right-hand page, i.e. they are both visible at the same time}
\footnotetext{Všechny příkazy nástroje FFmpeg používají navíc následující výstupní parametry:
\begin{mintedblock}
\inputminted{bash}{output-parameters.sh}
\end{mintedblock}
\noindent
Tyto parametry zajišťují vizuálně a auditivně bezztrátovou kompresi videa~\cite{ffmpeg-h265} a zvuku~\cite{ffmpeg-aac}.}
\begin{filecontents}[overwrite,nosearch,noheader]{example-minterpolate-frames-01.sh}
$ ffmpeg -i vstupní-video.mp4 -vf minterpolate obrázky-snímků/%06d.png
\end{filecontents}
\begin{filecontents}[overwrite,nosearch,noheader]{example-minterpolate-frames-02.sh}
$ ffmpeg -framerate 50 obrázky-snímků/%06d.png výstupní-video.mp4
\end{filecontents}
\addtocounter{footnote}{1}
\footnotetext{Při rychlém pohybu kamery nebo zabíraného objektu mohou při použití filtru \texttt{minterpolate} vznikat vizuální artefakty, především na okrajích snímků a objektů. Při zpracování přednášek Donalda Knutha na \acro{FI MU} jsem proto ze vstupního videa nejprve vytvořil obrázky snímků:
\begin{mintedblock}
\inputminted{bash}{example-minterpolate-frames-01.sh}
\end{mintedblock}
\noindent
Potom jsem obrázky snímků ručně prošel a smazal jsem ty, na kterých byly viditelné artefakty na osobě Donalda Knutha. Nakonec jsem z obrázků snímků vytvořil výsledné video:
\begin{mintedblock}
\inputminted{bash}{example-minterpolate-frames-02.sh}
\end{mintedblock}
\noindent
Video mělo proměnlivou frekvenci 25--50 snímků za sekundu podle počtu smazaných obrázků.}