\begingroup
\begin{subfigure}{\linewidth}
\input images/camcorder-minterpolate-yadif
\vspace{-10pt}
\caption{Při záznamu videa na digitální kameru dochází k zachycení mnoha snímků za vteřinu.}
\end{subfigure}
\setkeys{Gin}{width=28.74615pt,height=28.74615pt}%
\par\vspace{10pt}%
\begin{subfigure}[t]{0.49\linewidth}
\input images/frames/input-progressive-frames
\caption{U videa v neprokládaném formátu (vlevo nahoře) obsahují snímky obraz pro daný okamžik. U videa v prokládaném formátu (vpravo nahoře) obsahují snímky dva půlsnímky v polovičním rozlišení (vpravo) s obrazem ze dvou sousedních okamžiků.}
\end{subfigure}\hfill
\begin{subfigure}[t]{0.49\linewidth}
\input images/frames/input-interlaced-frames
\par\vspace{8pt}%
\input images/frames/input-uninterlaced-subframes-flat
\end{subfigure}
\par\vspace{10pt}%
\begin{subfigure}[t]{0.49\linewidth}
\begin{verbatim}
$ ffmpeg -i vstupní-video.mp4 \
>        -vf minterpolate     \
>        výstupní-video.mp4
\end{verbatim}
\input images/frames/output-minterpolate-frames
\caption{Filtr \texttt{minterpolate} dokáže u neprokládaných videí odhadnout obraz v mezilehlých okamžicích podle okolních snímků.\footnotemark}
\end{subfigure}\hfill
\begin{subfigure}[t]{0.49\linewidth}
\begin{verbatim}
$ ffmpeg -i vstupní-video.mp4 \
>        -vf yadif=send_field \
>        výstupní-video.mp4
\end{verbatim}
\input images/frames/output-yadif-frames
\caption{Filtr \texttt{yadif} dokáže u prokládaných videí zdvojnásobit rozlišení půlsnímků podle daného půlsnímku a jeho sousedů.}
\end{subfigure}
\endgroup