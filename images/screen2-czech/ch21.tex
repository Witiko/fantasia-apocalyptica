\chapter{Nové nebe a~nová země}
\verse{1}{110}{A viděl jsem nové \motif{nebe} a~novou zemi, neboť první nebe a~první země pominuly a~moře již vůbec nebylo.}
\verse{2}{127}{A viděl jsem od Boha z~nebe sestupovat \motif{svaté} město, nový \motif{Jeruzalém}, krásný jako nevěsta ozdobená pro svého ženicha.}
\verse{3}{171}{A slyšel jsem veliký hlas od trůnu: „Hle, \motif{příbytek} \motif{Boží} uprostřed lidí, Bůh bude přebývat mezi nimi a~oni budou jeho lid; on sám, jejich Bůh, bude s~nimi,}
\verse{4}{134}{a setře jim každou slzu z~\motif{očí}. A~smrti již nebude, ani žalu ani nářku ani bolesti už nebude – neboť co bylo, pominulo.“}
\verse{5}{129}{Ten, který seděl na trůnu, řekl: „Hle, všecko tvořím nové.“ A~řekl: „Napiš: Tato slova jsou věrná a~\motif{pravá}.“}
\verse{6}{135}{A dodal: „Již se vyplnila. Já jsem \motif{Alfa} i~\motif{Omega}, počátek i~konec. Tomu, kdo žízní, dám napít zadarmo z~pramene vody živé.}
\verse{7}{73}{Kdo \motif{zvítězí}, dostane toto vše; já mu budu Bohem a~on mi bude synem.}
\verse{8}{189}{Avšak zbabělci, nevěrní, nečistí, vrahové, cizoložníci, \motif{zaklínači}, modláři a~všichni lháři najdou svůj úděl v~jezeře, kde hoří oheň a~síra. To je ta druhá smrt.“}
\verse{9}{181}{A přistoupil jeden ze sedmi andělů, kteří měli těch sedm nádob a~v~nich připraveno sedm posledních pohrom, a~řekl mi: „Pojď, ukážu ti nevěstu, choť \motif{Beránkovu}.“}
\verse{10}{126}{Ve vytržení ducha mě vyvedl na velikou a~vysokou horu a~ukázal mi \motif{svaté} \motif{město} Jeruzalém, jak sestupuje z~nebe od Boha,}
\verse{11}{96}{zářící Boží \motif{slávou}; jeho jas jako nejdražší drahokam a~jako průzračný křišťál.}
\verse{12}{151}{Město mělo mohutné a~vysoké hradby, dvanáct bran střežených dvanácti \motif{anděly} a~na branách napsaná jména dvanácti pokolení synů Izraele.}
\verse{13}{95}{Tři brány byly na východ, tři brány na sever, tři brány na jih a~tři brány na západ.}
\verse{14}{131}{A hradby města byly postaveny na dvanácti základních kamenech a~na nich bylo dvanáct jmen dvanácti \motif{apoštolů} Beránkových.}
\verse{15}{91}{Ten, který se mnou mluvil, měl \coloredmotif{gold}{zlatou} míru, aby změřil \motif{město} i~jeho brány a~hradby.}
\verse{16}{169}{Město je vystaveno do čtverce: jeho délka je stejná jako šířka. Změřil to město, a~bylo to dvanáct tisíc měr. Jeho délka, šířka i~výška jsou stejné.}
\verse{17}{97}{Změřil i~hradbu, a~bylo to sto čtyřicet čtyři loket lidskou mírou, kterou použil anděl.}
\verse{18}{92}{Hradby jsou postaveny z~\motif{jaspisu} a~město je z~ryzího \coloredmotif{gold}{zlata}, zářícího jako křišťál.}
\verse{19}{137}{Základy hradeb toho města jsou samý drahokam: první základní kámen je \motif{jaspis}, druhý safír, třetí chalcedon, čtvrtý smaragd,}
\verse{20}{144}{pátý sardonyx, šestý karneol, sedmý chrysolit, osmý beryl, devátý \motif{topas}, desátý chrysopras, jedenáctý hyacint a~dvanáctý ametyst.}
\verse{21}{143}{A \motif{dvanáct} bran je z~\motif{dvanácti} perel, každá z~jediné perly. A~náměstí toho města je z~ryzího \coloredmotif{gold}{zlata} jako z~průzračného křišťálu.}
\verse{22}{90}{Avšak \motif{chrám} jsem v~něm nespatřil: Jeho chrámem je Pán Bůh všemohoucí a~Beránek.}
\verse{23}{126}{To město nepotřebuje ani \motif{slunce} ani měsíc, aby mělo světlo: září nad ním sláva Boží a~jeho světlem je Beránek.}
\verse{24}{80}{Národy budou žít v~jeho \motif{světle}; králové světa mu odevzdají svou slávu.}
\verse{25}{83}{Jeho brány zůstanou otevřeny, protože stále trvá den, a~noci tam už nebude.}
\verse{26}{48}{V něm se shromáždí \motif{sláva} i~čest národů.}
\verse{27}{126}{A nevstoupí tam nic nesvatého ani ten, kdo se rouhá a~lže, nýbrž jen ti, kdo jsou zapsáni v~Beránkově \motif{knize} života.}
