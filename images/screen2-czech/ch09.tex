\chapter{Hrůzy šesti polnic}
\verse{1}{119}{\motif{Zatroubil} pátý anděl. A~viděl jsem, jak \motif{hvězdě}, která spadla z~nebe na zem, byl dán klíč od jícnu propasti;}
\verse{2}{122}{otevřela jícen propasti a~vyvalil se \motif{dým} jako z~obrovské pece, a~tím dýmem se zatmělo slunce i~všechno ovzduší.}
\verse{3}{92}{Z dýmu se vyrojily \motif{kobylky} na zem; byla jim dána moc, jakou mají pozemští škorpióni.}
\verse{4}{140}{Dostaly rozkaz neškodit trávě na zemi ani žádné zeleni ani stromoví, jenom lidem, kteří nejsou označeni Boží \motif{pečetí} na čele.}
\verse{5}{147}{Ale nebyla jim dána moc, aby lidi zabíjely, nýbrž aby je po pět měsíců trýznily; byla to \motif{trýzeň}, jako když škorpión bodne člověka.}
\verse{6}{100}{V ty dny budou lidé hledat smrt, ale nenajdou ji, budou si přát zemřít, ale smrt se jim vyhne.}
\verse{7}{125}{Ty \motif{kobylky} vypadaly jako vyzbrojená válečná jízda; na hlavách měly něco jako \coloredmotif{gold}{zlaté} věnce, tvář měly jako lidé,}
\verse{8}{50}{hřívu jako vlasy žen, ale zuby měly jako lvi.}
\verse{9}{116}{Měly jakoby železné pancíře a~jejich \motif{křídla} hřmotila, jako když množství spřežení se řítí do boje.}
\verse{10}{93}{A měly ocasy jako škorpióni a~v~nich žihadla, aby jimi trýznily lidi po pět měsíců.}
\verse{11}{95}{Nad sebou měly krále, anděla \motif{propasti}, hebrejským jménem Abaddon – to znamená Hubitel.}
\verse{12}{62}{První ‚běda‘ pominulo, a~hle, už jsou tu \motif{dvě} další!}
\verse{13}{111}{\motif{Zatroubil} šestý anděl. Uslyšel jsem jakýsi hlas od čtyř rohů \coloredmotif{gold}{zlatého} oltáře, který je před Bohem.}
\verse{14}{131}{Ten hlas nařídil šestému andělu, držícímu polnici: „Rozvaž ty čtyři anděly, spoutané při veliké \motif{řece} Eufratu!“}
\verse{15}{118}{Tu byli rozvázáni ti čtyři andělé, připravení na \motif{hodinu}, den, měsíc a~rok, kdy mají pobít třetinu lidí.}
\verse{16}{86}{A jejich jízdních oddílů bylo dvě stě miliónů – slyšel jsem jejich počet.}
\verse{17}{190}{A ve vidění jsem spatřil koně a~na nich jezdce: pancíře měli ohnivé, rudě zářící a~žhnoucí sírou; hlavy koňů byly jako hlavy lvů a~z~jejich tlam šel oheň, \motif{dým} a~síra.}
\verse{18}{94}{Touto trojí pohromou – ohněm, \motif{dýmem} a~sírou ze svých tlam – usmrtili třetinu lidí.}
\verse{19}{139}{Vražedná moc těch koňů je v~jejich tlamách a~v~jejich ocasech; jejich ocasy jsou jako hadi a~hlavy těch hadů přinášejí zhoubu.}
\verse{20}{248}{A přesto se ostatní lidé, kteří v~těch pohromách nezahynuli, \motif{neodvrátili} od výtvorů svých rukou; nepřestali se klanět démonům a~modlám ze \coloredmotif{gold}{zlata}, stříbra, mědi, kamene i~dřeva, které jsou slepé, hluché a~nemohou se pohybovat;}
\verse{21}{78}{neodvrátili se od svých vražd ani \motif{čarování}, necudností ani krádeží.}
