\chapter{Hrůzy šesti pečetí}
\verse{1}{152}{Tu jsem viděl, jak Beránek \motif{rozlomil} první ze sedmi pečetí, a~slyšel jsem, jak jedna z~těch čtyř bytostí řekla hromovým hlasem: „Pojď!“}
\verse{2}{100}{A hle, bílý \motif{kůň}, a~na něm jezdec s~lukem; byl mu dán věnec dobyvatele, aby vyjel a~dobýval.}
\verse{3}{103}{Když Beránek \motif{rozlomil} druhou pečeť, slyšel jsem, jak druhá z~těch bytostí řekla: „Pojď!“}
\verse{4}{142}{A vyjel druhý \motif{kůň}, ohnivý, a~jeho jezdec obdržel moc odejmout zemi pokoj, aby se všichni navzájem vraždili; byl mu dán veliký meč.}
\verse{5}{155}{Když Beránek \motif{rozlomil} třetí pečeť, slyšel jsem, jak třetí z~těch bytostí řekla: „Pojď!“ A~hle, \motif{kůň} černý, a~jezdec měl v~ruce váhy.}
\verse{6}{155}{A z~kruhu těch čtyř bytostí jsem slyšel \motif{hlas}: „Za denní mzdu jen mírka pšenice, za denní mzdu tři mírky ječmene. Olej a~víno však nech!“}
\verse{7}{93}{A když Beránek \motif{rozlomil} čtvrtou pečeť, slyšel jsem hlas čtvrté bytosti: „Pojď!“}
\verse{8}{187}{A hle, \motif{kůň} sinavý, a~jméno jeho jezdce Smrt, a~svět mrtvých zůstával za ním. Těm jezdcům byla dána moc, aby čtvrtinu země zhubili mečem, hladem, morem a~dravými šelmami.}
\verse{9}{139}{Když Beránek \motif{rozlomil} pátou pečeť, spatřil jsem pod oltářem ty, kdo byli zabiti pro slovo Boží a~pro svědectví, které vydali.}
\verse{10}{134}{A křičeli velikým hlasem: „Kdy už, Pane \motif{svatý} a~věrný, vykonáš \motif{soud} a~za naši \coloredmotif{blood}{krev} potrestáš ty, kdo bydlí na zemi?“}
\verse{11}{192}{Tu jim všem bylo dáno bílé roucho a~bylo jim řečeno, aby měli strpení ještě krátký čas, dokud jejich počet nedoplní spoluslužebníci a~bratří, kteří budou zabiti jako oni.}
\verse{12}{136}{A hle, když \motif{rozlomil} šestou pečeť, nastalo veliké \motif{zemětřesení}, slunce zčernalo jako smuteční šat, měsíc úplně \coloredmotif{blood}{zkrvavěl}}
\verse{13}{103}{a nebeské \motif{hvězdy} začaly padat na zem, jako když fík zmítaný vichrem shazuje své pozdní plody,}
\verse{14}{107}{nebesa zmizela, jako když se zavře \motif{kniha}, a~žádná hora a~žádný ostrov nezůstaly na svém místě.}
\verse{15}{153}{\motif{Králové} země i~velmoži a~vojevůdci, boháči a~mocní – jak \motif{otrok}, tak svobodný, všichni prchali do hor, aby se ukryli v~jeskyních a~skalách,}
\verse{16}{142}{a volali k~horám a~skalám: „Padněte na nás a~\motif{skryjte} nás před tváří toho, který sedí na trůnu, a~před hněvem Beránkovým!“}
\verse{17}{64}{Neboť přišel veliký den jeho \motif{hněvu}; kdo bude moci obstát?}
